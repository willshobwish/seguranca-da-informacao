% Define o tipo de documento como "article", com fonte de 12pt
% A opção 'nodisplayskipstretch' evita que o LaTeX altere o espaçamento ao redor de fórmulas exibidas
\documentclass[12pt,nodisplayskipstretch]{article}

% Importa o arquivo com as configurações gerais do projeto (pacotes, margens, fonte etc.)
\usepackage{config/config}

% Importa comandos personalizados definidos pelo usuário, como macros para título, nome, orientador etc.
\usepackage{config/commands}

% Importa redefinições de comandos padrão do LaTeX, para personalizar o comportamento de comandos existentes
\usepackage{config/renew-commands}

\begin{document} % Início do conteúdo do documento

% ----------------------------
% SEÇÕES PRÉ-TEXTUAIS
% ----------------------------

% Importa a folha de rosto (com título, autor, orientador etc.)
\import{sections/pre-textual/}{0-folha-de-rosto.tex}

% Importa a folha de aprovação (assinaturas da banca)
\import{sections/pre-textual/}{1-folha-de-aprovacao.tex}

% Importa o resumo em português
\import{sections/pre-textual/}{2-resumo.tex}

% Importa o resumo em inglês (abstract)
\import{sections/pre-textual/}{3-resumo-ingles.tex}

% Importa listas automáticas (figuras, tabelas, siglas etc.)
\import{sections/pre-textual/}{4-listas.tex}

% Importa o sumário (índice do conteúdo)
\import{sections/pre-textual/}{5-sumario.tex}

% ----------------------------
% SEÇÕES TEXTUAIS
% ----------------------------

% Importa a introdução do trabalho
\import{sections/textual/}{6-introducao.tex}

% Importa a seção de materiais e métodos (ou metodologia)
\import{sections/textual/}{7-materiais-e-metodos.tex}

% Importa a seção de resultados e discussão
\import{sections/textual/}{8-resultados-e-discussao.tex}

% Importa a conclusão do trabalho
\import{sections/textual/}{9-conclusoes.tex}

% Importa os agradecimentos (opcional, mas comum)
\import{sections/textual/}{11-agradecimentos.tex}

% Importa as referências bibliográficas
\import{sections/textual/}{10-referencias.tex}

\end{document} % Fim do conteúdo do documento
